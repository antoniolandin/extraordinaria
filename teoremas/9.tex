\begin{thm}
    El centro de un grupo es un subgrupo y es normal.
\end{thm}

\begin{proof}
    El centro $Z(G)$ es un subgrupo de $G$
    \begin{enumerate}
        \item El centro $Z(G)$, al mantener la operación de $G$, seguirá cumpliendo la propiedad asociativa.
        \item El centro mantiene el neutro:
        \begin{equation}
            \forall x \in G: e \cdot x = x \cdot e \implies e \in Z(G)
        \end{equation}
    \item Si $x$ está en el centro entonces su inverso $x^{-1}$ también:
        \begin{equation}
            \begin{split}
            & x \in Z(G) \implies \forall g \in G : g \cdot x^{-1} = x\cdot x^{-1}\cdot g\cdot x^{-1} =\\
            & x^{-1} \cdot g \cdot x \cdot x^{-1} = x^{-1} \implies x^{-1} \in Z(G) 
            \end{split}
        \end{equation}
    \end{enumerate}
\end{proof}

\begin{proof}
    El centro $Z(G)$ es un subgrupo normal
    \begin{equation}
        \begin{split}
            & \forall h \in Z(G), g \in G: g \cdot h \cdot g^{-1}  = g \cdot g^{-1} \cdot h = h \implies\\
            & g H g^{-1} = H \implies Z(G) \text{ subgrupo normal}  
        \end{split}
    \end{equation}
\end{proof}
