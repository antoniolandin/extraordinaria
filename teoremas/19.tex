\begin{thm}
    El núcleo de un homomorfismo es un subgrupo y es normal.
\end{thm}

\begin{proof}
    La operación está contenida en $\text{Ker}f$:
    \begin{equation}
        \forall a,b \in \text{Ker} f: f(a\cdot b) = f(a) \cdot f(b) = e \cdot e = e \implies a\cdot b \in \text{Ker} f
    \end{equation}

    \begin{enumerate}
        \item El subgrupo del nucleo al mantener la operación, sigue manteniendo también la propiedad asociativa.
        \item El neutro pertenece al nucleo:
        \begin{equation}
            f(e) = e \implies e \in \text{Ker} f
        \end{equation}
        \item Si $x \in \text{Ker} f$ entonces su inverso también.
        \begin{equation}
            \begin{split}
                & x \in \text{Ker} f : f(e) = e \implies f(x \cdot x^{-1}) = e \implies f(x) \cdot f(x^{-1}) = e \implies\\
                & e \cdot f(x^{-1}) = e \implies f(x^{-1}) = e \implies x^{-1} \in \text{Ker} f 
            \end{split}
        \end{equation}
    \end{enumerate}

    Por lo tanto, el nucleo es un homomorfismo.

    Además, es normal ya si llamamos $H = \text{Ker} f$ que se cumple que $\forall g \in G: g H g^{-1} = H$.
    \begin{equation}
        f(gHg^{-1}) = f(g) \cdot f(H) \cdot f(g^{-1}) = f(g) \cdot f(g^{-1}) = f(g) \cdot f^{-1}(g) = e \implies g H g^{-1} = H  
    \end{equation}
\end{proof}
