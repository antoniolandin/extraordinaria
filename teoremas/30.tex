\begin{thm}
    Los polinomios irreducibles en reales tienen grado uno o dos.
\end{thm}

\begin{proof}
    Por el teorema fundamental del álgebra sabemos que un polinomio tiene $n$ raíces en los complejos. Además, las raíces complejas vienen en pares de conjugados.

    Los polinomios de grado 1 son siempre irreducibles en los reales.

    Un polinomio de grado 2 puede o tener 2 raíces reales (reducible) o dos complejas (irreducible en los reales).

    Para el resto de grados siempre va a tener o 1 raíz real o 4 o más raíces complejas las cuales como mínimo formarán 2 polinomios reales (una factorización que hará al polinomio reducible).
\end{proof}
