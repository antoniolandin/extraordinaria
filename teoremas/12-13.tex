\begin{thm}[Teorema de Lagrange]
    \label{thm:lagrange}
    El orden del grupo es múltipo del orden del subgrupo
    \begin{equation}
        H \subseteq G \implies |G| = |H| \cdot n \quad n \in \mathbb{N}
    \end{equation}
\end{thm}

\begin{proof}
    Dado un grupo $G$ con un subgrupo $H$ y con clases laterales $x_iH$, por el teorema \ref{thm:clases-laterales} sabemos que las clases laterales no tienen intersección. Como tienen que abarcar a todo el grupo $G$ y no tienen intersección, entonces necesariamente el orden de $G$ será múltiplo del orden de las clases laterales.
    \begin{equation}
        \sum_{i=1}^{\text{índice}} |x_iH| = |G| \implies |G| = |x_iH| \cdot n \quad n \in \mathbb{N}
    \end{equation}
    Las clases laterales $x_iH$ tienen el mismo orden que $H$, ya que se construyen operando $x_i$ con los elementos de $H$, entonces el orden de $G$ es múltiplo del orden de $H$.
    \begin{equation}
        |x_iH| = |H| \implies |G| = |H| \cdot n \quad n \in \mathbb{N}
    \end{equation}
\end{proof}

\begin{cor}
    El orden del grupo es múltiplo del orden de un elemento
    \begin{equation}
        x \in G: x^{k} = e \implies |G| = k \cdot n \quad n \in \mathbb{N}
    \end{equation}
    \label{cor:lagrange}
\end{cor}
