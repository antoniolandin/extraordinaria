\begin{thm}
    Un ideal $I$ que posee al neutro multiplicativo coincide con el anillo que lo contiene.
    \begin{equation}
        1 \in I \subseteq R \implies I = R
    \end{equation}
    \label{thm:ideal-neutro-multiplicativo}
\end{thm}

\begin{proof}
    Como $1 \in I$ entonces, por definición: 
    \begin{equation}
        \forall x \in R: 1 \cdot x \in I \implies \forall x \in R: x \in I \implies I = R
    \end{equation}
\end{proof}
