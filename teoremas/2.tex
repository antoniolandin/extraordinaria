\begin{thm}
    En cada grupo solo puede existir un elemento neutro.
    \label{thm:neutro}
\end{thm}

\begin{proof}
    Sea el grupo $G = \{X, \cdot\}$ asumamos que tenga dos elementos neutros $e_1$ y $e_2$, entonces
    \begin{equation}
        \begin{cases}
            x \cdot e_{1} = x\\
            x \cdot e_{2} = x
        \end{cases} \implies x \cdot e_{1} = x \cdot e_{2} \iff x^{-1} \cdot x \cdot e_{1} = x^{-1} \cdot x \cdot e_{2} \iff e \cdot e_{1} = e \cdot e_{2} \iff e_{1} = e_{2}  
    \end{equation}
\end{proof}
