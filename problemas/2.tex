\subsection{Averiguar si algún conjunto es semigrupo, monoide o grupo.}

\begin{enumerate}
    \item El conjunto de cadenas de símbolos (\textit{"string"}) con la concatenación como operación binaria.

    \begin{enumerate}
        \item Cumple con la propiedad asociativa:
        \begin{equation}
            (\text{"a"} + \text{"b"}) + \text{"c"} = \text{"a"} + (\text{"b"} + \text{"c"}) = \text{"abc"}
        \end{equation}

        \item El elemento neutro es la cadena vacía ($\text{""}$):
        \begin{equation}
            \text{"a"} + \text{""} = \text{"a"}
        \end{equation}

        \item No existe elemento inverso
    \end{enumerate}

    Por lo tanto, se trata de un monoide.

    \item El conjunto $\{\text{Hija, Madre, Abuela}\}$ con la operación \textit{"La mayor de las dos"}.

    \begin{enumerate}
        \item Cumple con la propiedad asociativa, ya que el resultado siempre será el mayor de todos:
        \begin{equation}
            \begin{split}
                & \text{Hija} * (\text{Madre} * \text{Abuela}) = \text{Hija} * \text{Abuela} = \text{Abuela} =\\
                & (\text{Hija} * \text{Madre}) * \text{Abuela} = \text{Madre} * \text{Abuela} = \text{Abuela}
            \end{split}
        \end{equation}

        \item El elemento neutro es $e = \text{Hija}$:
        \begin{equation}
            \text{Hija} * \text{Hija} = \text{Hija} \quad \text{Madre} * \text{Hija} = \text{Madre} \quad \text{Abuela} * \text{Hija} = \text{Abuela}
        \end{equation}

        \item No existe el elemento inverso.
    \end{enumerate}
    
    Por lo tanto, se trata de un monoide.

    \item El conjunto $\{\text{Giro de } 0^{\circ}, \text{Giro de } 120^{\circ}, \text{Giro de } 240^{\circ} \}$ con la composición de giros.

    \begin{enumerate}
        \item Cumple con la propiedad asociativa, ya que no importa que giro aplicar primero, al final saldrán los mismo grados.
        \begin{equation}
            (0^{\circ} * 120^{\circ} ) * 240^{\circ} = 120^{\circ} * 240^{\circ} = 0^{\circ} = 0^{\circ} * (120^{\circ} * 240^{\circ} ) = 0^{\circ} * 0^{\circ} = 0^{\circ}        
        \end{equation}
        \item El elemento neutro es el giro de 0 grados $e = 0^{\circ}$
        \item Cada elemento tiene un inverso:
        \begin{equation}
            \begin{split}
                & 0^{\circ} * 0^{\circ} = 0^{\circ} \implies (0^{\circ})^{-1} = 0^{\circ}\\
                & 120^{\circ} * 240^{\circ} = 0^{\circ} \implies (120^{\circ})^{-1} = 240^{\circ}\\
                & 240^{\circ} * 120^{\circ} = 0^{\circ} \implies (240^{\circ})^{-1} = 120^{\circ}     
            \end{split}
        \end{equation}
    \end{enumerate}

    Por lo tanto, se trata de un grupo.

    \item Los enteros positivos pares con la operación suma.

    \begin{enumerate}
        \item Al tratarse de la suma de enteros, cumple la propiedad asociativa.
        \item El neutro 0, no pertence al conjunto, por lo tanto no existe un neutro.
    \end{enumerate}

    Se trata de un semigrupo.

    \item Los enteros positivos pares más el cero con la operación suma.

    \begin{enumerate}
        \item Al tratarse de la suma de enteros, cumple la propiedad asociativa.
        \item El neutro es el 0, ya que se trata de la suma.
        \item Al no tener negativos, el único elemento con inverso es el 0.
    \end{enumerate}

    Por lo tanto, se trata de un monoide.

    \item $\mathbb{Q}$ con la suma.

    \begin{enumerate}
        \item Al tratarse de la suma de racionales, cumple la propiedad asocitativa.
        \item El neutro es $0 \in \mathbb{Q}$, ya que se trata de la suma.
        \item Cada elemento $a \in \mathbb{Q}$ tiene inverso $-a \in \mathbb{Q}$
    \end{enumerate}

    Por lo tanto, se trata de un grupo.

    \item $\mathbb{Q}$ con la multiplicación.

    \begin{enumerate}
        \item Al tratarse del producto de racionales, cumple la propiedad asociativa.
        \item El 1 es el elemento neutro.
        \item Todos los elementos tienen inverso excepto el 0.
    \end{enumerate}

    Por lo tanto, se trata de un monoide. Para que fuese un grupo, habría que eliminar al 0 del conjunto.

    \item $\mathbb{R}\setminus \{0\}$ con la división.

    \begin{enumerate}
        \item La división no cumple con la propiedad asociativa.
    \end{enumerate}

    Por lo tanto, no es ni semigrupo, ni monoide, ni grupo.

    \item $\mathbb{Z}$ con la suma.

    \begin{enumerate}
        \item Al tratarse de la suma de enteros, cumple con la propiedad asociativa.
        \item El 0 es el elemento neutro.
        \item Cada elemento $a \in \mathbb{Z}$ tiene inverso $-a \in \mathbb{Z}$.
    \end{enumerate}
    
    Por lo tanto, se trata de un grupo.

    \item $\mathbb{R}\setminus \{0\}$ con la operación $a \cdot b = 3ab$
    
    \begin{enumerate}
        \item Como en la operación solo intervienen productos, cumple con la propiedad asociativa.
        \item El elemento inverso es $e=\frac{1}{3}$:
            \begin{equation}
                a \cdot \frac{1}{3} = 3a\cdot \frac{1}{3} = a
            \end{equation}
        \item El inverso de $a$ es $a^{-1} = \frac{1}{9a}$:
            \begin{equation}
                a \cdot \frac{1}{9a} = 3a \cdot \frac{1}{9a} = \frac{1}{3} = e
            \end{equation}
    \end{enumerate}

    Por lo tanto, es un grupo.

    \item $\mathbb{R}\setminus \{-1\}$ con la operación $a \cdot b = a + b + ab$
    \begin{enumerate}
        \item Como en la operación solo intervienen sumas y productos, cumple con la propiedad asociativa.
        \item El elemento neutro es el 0:
        \begin{equation}
            a \cdot 0 = a + 0 + a\cdot 0 = a
        \end{equation}
        \item El inverso de $a$ es $a^{-1} = -\frac{a}{1+a}$:
        \begin{equation}
            a + a^{-1} + aa^{-1} = 0 \iff a^{-1}(1 + a) + a = 0 \iff a^{-1} = -\frac{a}{1+a}
        \end{equation}
    \end{enumerate}

    Por lo tanto, se trata de un grupo.
\end{enumerate}
