\begin{ce}
    $\mathbb{Q}[\sqrt{2}]$ es isomorfo a $\mathbb{Q}[\sqrt{5}]$
\end{ce}

\begin{proof}
    Como estamos diciendo que es isomorfo, el neutro multiplicativo se tiene que conservar:
    \begin{equation}
        f(1) = 1 \implies f(a) = a
    \end{equation}

    Entonces, necesariamente $f(\sqrt{2}) = a\sqrt{5}$.
    \begin{equation}
        2 = f(2) = f(\sqrt{2}\sqrt{2}) = f(\sqrt{2})f(\sqrt{2}) = (a\sqrt{5})^2 = 5a^2 \implies a^2 = \frac{2}{5}
    \end{equation}

    Pero, $a^2 \neq \frac{2}{5}$ (misma demostración que $\sqrt{2}$ irracional).
\end{proof}
