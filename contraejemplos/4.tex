\begin{ce}
    Un polinomio de grado $n$ no puede tener más de $n$ raíces.
\end{ce}

\begin{proof}
    En el anillo $\mathbb{Z}_{6}$ el polinomio $P(x)=(x-2)(x-3)$ tiene más de 2 raíces:
    \begin{equation}
        \begin{split}
            & P(0) = (0-2)(0-3) = (-2)(-3) = 6 \equiv 0 \mod 6\\
            & P(2) = (2-2)(2-3) = 0\cdot (-3) = 0\\
            & P(3) = (3-2)(3-3) = 1 \cdot 0 = 0\\
            & P(5) = (5-2)(5-3) = 3 \cdot 2 = 6 \equiv 0 \mod 6
        \end{split}
    \end{equation}
\end{proof}
