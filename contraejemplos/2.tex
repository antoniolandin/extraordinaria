\begin{ce}
    La imagen y el núcleo de un homomorfismo de anillos son ideales.
\end{ce}

\begin{proof}
    Es cierto que el núcleo es siempre un ideal, pero la imagen solo si se trata de un homomorfismo suprayectivo. Si consideramos por ejemplo el homomorfismo $f:\mathbb{R} \to M_{2}(\mathbb{R})$ tal que:
    \begin{equation}
        f(x) = \begin{pmatrix}
            x & 0\\
            0 & 0
        \end{pmatrix} 
    \end{equation}

    Entonces $\text{Im} f$ ya no es un ideal ya que la multiplicación de un elemento de $M_{2}(\mathbb{R})$ con la imágen se puede salir de $\text{Im} f$:
    \begin{equation}
        \begin{pmatrix}
            x & 0\\
            0 & 0
        \end{pmatrix} 
        \cdot
        \begin{pmatrix}
            1 & 1\\
            0 & 0
        \end{pmatrix} =
        \begin{pmatrix} 
            x & x\\
            0 & 0
        \end{pmatrix} \not\in \text{Im}f
    \end{equation}
\end{proof}
