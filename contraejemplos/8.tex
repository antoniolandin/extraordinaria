\begin{ce}
    El neutro multiplicativo del subanillo coincide con la del anillo principal.
\end{ce}

\begin{proof}
    $\{0,2,4\} \subset \mathbb{Z}_{6}$ tiene como neutro multiplicativo al $4$:
    \begin{equation}
        \begin{split}
            & 0 \cdot 4 = 0\\
            & 2 \cdot 4 = 8 \equiv 2 \mod 4\\
            & 4 \cdot 4 = 16 \equiv 4 \mod 4
        \end{split}
    \end{equation}
    Pero en $\mathbb{Z}_{6}$ el neutro multiplicativo es el $1$.
\end{proof}
