\documentclass[a4paper]{article}

\usepackage[spanish,es-noshorthands]{babel}
\usepackage{fullpage}
\usepackage{amsfonts}
\usepackage{amsmath}
\usepackage{amsthm, thmtools, cleveref, etoolbox}
\usepackage{amssymb}
\usepackage{xparse}

\allowdisplaybreaks
\setlength\parindent{0pt}
\theoremstyle{plain}

% Theorems in TOC 
\makeatletter
\newcommand{\HelperDeclareTheorem}[4]{%
  \declaretheorem[
    numberlike=subsection,
    name=#3,
    refname={#1, #2},
    Refname={#3, #4},
    postheadhook={%
      \addcontentsline{toc}{subsection}{\protect\numberline{\thesubsection} #3 \ifdefempty{\thmt@optarg}{}{(\thmt@optarg)}}%
    }
  ]{%
    #1%
  }%
}
\makeatother
\HelperDeclareTheorem{thm}{thms}{Teorema}{Teoremas}
\HelperDeclareTheorem{lem}{lems}{Lema}{Lemas}
\HelperDeclareTheorem{cor}{cors}{Colorario}{Colorarios}
\HelperDeclareTheorem{defi}{defis}{Definición}{Definiciones}
\HelperDeclareTheorem{ce}{ces}{Contraejemplo}{Contraejemplos}

% Anotattion command
\newcommand{\annoterel}[2]{%
  \overset{%
    \substack{\hidewidth\text{#1}\hidewidth\\\downarrow}%
  }{#2}%
}

\author{Antonio Cabrera Landín}
\title{Extraordinaria Estructuras Algebraicas}
\date{\today}

\begin{document}
    \maketitle
    \tableofcontents
    \section{Teoremas}
\begin{thm}[Propiedad cancelativa]
    Para todo grupo $G=\{X,\cdot\}$ se cumple
    \begin{equation}
        a \cdot c = b \cdot c \implies a = b
    \end{equation}
    \begin{equation}
        c \cdot a = c \cdot b \implies a = b
    \end{equation}
\end{thm}

\begin{proof}
    \begin{equation}
        a \cdot c = b \cdot c \iff a \cdot c \cdot c^{-1} = b \cdot c \cdot c^{-1} \iff a \cdot e = b \cdot e \iff a = b  
    \end{equation}
    \begin{equation}
        c \cdot a = c \cdot b \iff c^{-1} \cdot c \cdot a = c^{-1} \cdot c \cdot b \iff e \cdot a = e \cdot b \iff a = b  
    \end{equation}
\end{proof}

\begin{thm}
    En cada grupo solo puede existir un elemento neutro.
    \label{thm:neutro}
\end{thm}

\begin{proof}
    Sea el grupo $G = \{X, \cdot\}$ asumamos que tenga dos elementos neutros $e_1$ y $e_2$, entonces
    \begin{equation}
        \begin{cases}
            x \cdot e_{1} = x\\
            x \cdot e_{2} = x
        \end{cases} \implies x \cdot e_{1} = x \cdot e_{2} \iff x^{-1} \cdot x \cdot e_{1} = x^{-1} \cdot x \cdot e_{2} \iff e \cdot e_{1} = e \cdot e_{2} \iff e_{1} = e_{2}  
    \end{equation}
\end{proof}

\begin{thm}
    Dado un grupo $G = \{X, \cdot \}$, si para cualquier elemento $a \in G$ se cumple que $a \cdot a = e$, entonces $G$ es abeliano.
    \begin{equation}
        \forall a \in G : a\cdot a = e \implies \forall a,b \in G : a \cdot b = b \cdot a
    \end{equation}
\end{thm}

\begin{proof}
    \begin{equation}
        a \cdot b \in G \implies (ab)^2 = ab\cdot ab = e
    \end{equation}

    \begin{equation}
        \begin{split}
            & aabb = e \cdot e = e = (ab)^2 \implies a^{-1}(aabb) = a^{-1}(abab) \iff\\
            & abb = bab \iff (abb)b^{-1} = (bab)b^{-1} \iff ab = ba 
        \end{split}
    \end{equation}
\end{proof}

\begin{thm}
    Dado un grupo $G = \{X, \cdot\}$, si para todo $a,b \in G$ se cumple que $(a\cdot b)^2 = a^2 \cdot b^2$, entonces $G$ es abeliano
    \begin{equation}
        \forall a,b \in G: (a\cdot b)^2 = a^2 \cdot b^2 \implies \forall a,b \in G: a \cdot b = b \cdot a
    \end{equation}
\end{thm}

\begin{proof}
    \begin{equation}
        \begin{split}
        & (a \cdot b)^2 = a^2 \cdot b^2 \implies abab = aabb \iff a^{-1}(abab) = a^{-1}(aabb) \iff\\
        & bab = abb \iff (bab)b^{-1} = (abb)b^{-1} \iff ba = ab  
        \end{split}
    \end{equation}
\end{proof}

\begin{thm}
    $\mathbb{Z}_n$ tiene $\varphi(n)$ generadores
\end{thm} 

\begin{thm}
    Dado un grupo $G = \{X, \cdot\}$, para todos sus elementos $a\in G$ el inverso del inverso de $a$ es $a$.
    \begin{equation}
        \forall a \in G : (a^{-1})^{-1} = a 
    \end{equation}
\end{thm}

\begin{proof}
    \begin{equation}
        (a^{-1})^{-1} \cdot a^{-1} = e \iff  (a^{-1})^{-1} \cdot a^{-1} \cdot a = e \cdot a \iff (a^{-1})^{-1} = a
    \end{equation} 
\end{proof}

\begin{lem}
    Dado un grupo $G = \{X, \cdot\}$, para cada pareja $a,b \in G$ se cumple que $(a\cdot b)^{-1}=b^{-1}a^{-1}$
    \label{lem:inverso}
\end{lem}

\begin{proof}
    \begin{equation}
        \begin{split}
            & (a \cdot b)^{-1} (a \cdot b) = e \iff (a\cdot b)^{-1}(a\cdot b)\cdot b^{-1} = e \cdot b^{-1} \iff\\
            & (a\cdot b)^{-1}a = b^{-1} \iff (a\cdot b)^{-1}a\cdot a^{-1} = b^{-1}a^{-1} \iff\\
            & (a\cdot b)^{-1} = b^{-1} a^{-1}
        \end{split}
    \end{equation}
\end{proof}

\begin{thm}
    Dado un grupo $G = \{X, \cdot\}$, para cada pareja $a,b \in G$ el orden de $ab$ es el mismo que el de $ba$
    \begin{equation}
        \forall a,b \in G: (a\cdot b)^{k} = e \iff (b \cdot a)^{k} = e  
    \end{equation}
\end{thm}

\begin{proof}
    \begin{equation}
        \begin{split}
        & (ab)^{k} = ab\cdot ab\cdots ab = a \cdot (ba \cdots ba) \cdot b = a \cdot (ba)^{k-1} \cdot b = e \iff\\
        & a^{-1}\cdot a\cdot (ba)^{k-1}\cdot b = a^{-1} \iff (ba)^{k-1}\cdot b = a^{-1} \iff\\
        & (ba)^{k-1}\cdot b \cdot b^{-1}  = a^{-1}b^{-1} \iff (ba)^{k-1} = a^{-1}b^{-1} \annoterel{\text{lema} \ref{lem:inverso}}{=} (ba)^{-1} \iff\\
        & (ba)^{k-1}(ba) = (ba)^{-1}(ba) \iff (ba)^{k} = e   
        \end{split}
    \end{equation}
\end{proof}

\begin{thm}[Teorema de Caley]
    Todo grupo $G$ es isomorfo a un subgrupo de las biyecciones de $G$, $B(G)$, que es el grupo simétrico.
\end{thm}

\begin{proof}
    Consideremos las acciones de un elemento $\varphi_g(x) = gx$. Primero demostraremos que son biyectivas:
    
    \begin{itemize}
        \item Las acciones de un elemento son inyectivas ya que si consideramos $x \neq y$ entonces:
            \begin{equation}
                \begin{cases}
                    \varphi_g(x) = gx\\
                    \varphi_g(y) = gy
                \end{cases} 
            \end{equation}

            Por lo tanto:
            \begin{equation}
                x \neq y \implies gx \neq gy \implies \varphi_g(x) \neq \varphi_g(y)
            \end{equation}

        \item Las acciones son sobreyectivas ya que para cualquier imágen $x \in \text{Im}f$ existe la preimagen $g^{-1}x$:
            \begin{equation}
                \varphi(g^{-1}x) = gg^{-1}x = x \in \text{Im}f 
            \end{equation}

            Por lo tanto como las acciones de un elemento son sobreyectivas e inyectivas, son también biyectivas.
    \end{itemize}

    A continuación, demostraremos las biyecciones de $G$, $B(G)$, es un grupo:

    \begin{itemize}
        \item Está correctamente definido:

            $\varphi_a \circ \varphi_b(x) = \varphi_a(\varphi_b(x))=\varphi_a(bx) = abx \implies \varphi_a \circ \varphi_b = \varphi_{ab}$
        \item Como la operación es la composición de funciones, cumple con la propiedad asociativa.

        \item El neutro es $\varphi_e \in B(G)$:
        \begin{equation}
            \varphi_{a} \circ \varphi_{e} = \varphi_{a \cdot e} = \varphi_{a}
        \end{equation}

        \item El inverso de $\varphi_a \in B(G)$ es $\varphi_{a^{-1}} \in B(G)$:
        \begin{equation}
            \varphi_a \circ \varphi_{a^{-1}} = \varphi_{a \cdot a^{-1}} = \varphi_e
        \end{equation}
    \end{itemize}
    
    Ahora falta demostrar que es isomorfo al grupo original.
    \begin{equation}
        G \cong B(G)
    \end{equation}

    Consideremos el morfismo $f:G \to B(G)$ tal que $f(g) = \varphi_g$. Demostraremos primero que se trata de un homomorfismo:
    \begin{equation}
        f(a \cdot b) = \varphi_{ab} = \varphi_a \circ \varphi_b \impliedby \varphi_a(\varphi_b(x)) = \varphi_a(bx) = abx = \varphi_{ab}(x)
    \end{equation}

    Por último, demostraremos que este homomorfismo es biyectivo:

    \begin{itemize}
        \item El homomorfismo es inyectivo: Supongamos $a,b \in G : a\neq b$:
            \begin{equation}
                a \neq b \implies \forall x \in G: ax \neq bx \iff \varphi_a(x) \neq \varphi_b(x) \iff \varphi_a \neq \varphi_b
            \end{equation}
        \item El homomorfismo es suprayectivo, ya que para cada imagen $\varphi_a \in B(G)$ existe la preimagen $a \in G$.
    \end{itemize}

    Como el homomorfismo es biyectivo, hemos encontrador un isomorfismo de $G$ a $B(G)$, por lo tanto, cualquier grupo $G$ es isomorfo a un subgrupo de las biyecciones $B(G)$.
\end{proof}

\begin{thm}
    El centro de un grupo es un subgrupo y es normal.
\end{thm}

\begin{proof}
    El centro $Z(G)$ es un subgrupo de $G$
    \begin{enumerate}
        \item El centro $Z(G)$, al mantener la operación de $G$, seguirá cumpliendo la propiedad asociativa.
        \item El centro mantiene el neutro:
        \begin{equation}
            \forall x \in G: e \cdot x = x \cdot e \implies e \in Z(G)
        \end{equation}
    \item Si $x$ está en el centro entonces su inverso $x^{-1}$ también:
        \begin{equation}
            \begin{split}
            & x \in Z(G) \implies \forall g \in G : g \cdot x^{-1} = x\cdot x^{-1}\cdot g\cdot x^{-1} =\\
            & x^{-1} \cdot g \cdot x \cdot x^{-1} = x^{-1} \implies x^{-1} \in Z(G) 
            \end{split}
        \end{equation}
    \end{enumerate}
\end{proof}

\begin{proof}
    El centro $Z(G)$ es un subgrupo normal
    \begin{equation}
        \begin{split}
            & \forall h \in Z(G), g \in G: g \cdot h \cdot g^{-1}  = g \cdot g^{-1} \cdot h = h \implies\\
            & g H g^{-1} = H \implies Z(G) \text{ subgrupo normal}  
        \end{split}
    \end{equation}
\end{proof}

\begin{thm}
    Todos los subgrupos de índice 2 son normales
\end{thm}

\begin{proof}
    Si un subgrupo tiene índice 2, significa que solo tiene dos clases laterales izquierdas y 2 clases laterales derechas.

    Tendremos dos clases laterales izquierdas, $aH$ y $bH$. Como $e$ estará en una de las dos, podemos hablar de $H$ y $xH$. Lo mismo ocurre con las clases laterales derechas, tendremos $H$ y $Hx$
    \begin{equation}
        \begin{cases}
            H \cup  xH = G\\
            H \cup Hx = G
        \end{cases} \implies xH = Hx \implies H \text{ es normal}
    \end{equation}
\end{proof}

\begin{lem}
    Si $a \in xH$ entonces $xH = aH$
\end{lem}

\begin{proof}
    \begin{equation}
        a \in xH \implies \exists h \in H : a = xh \iff aH = xhH \implies aH = xH
    \end{equation}
    \label{lem:clases-laterales}
\end{proof}

\begin{thm}
    Las clases laterales izquierdas o coinciden o no tienen intersección.
    \label{thm:clases-laterales}
\end{thm}


\begin{proof}
    Dado un grupo $G=\{X, \cdot\}$ con un subgrupo $H \subseteq G$.
    Supongamos que $\exists a \in xH$ que además $a \in yH$, entonces:
    \begin{equation}
        \begin{split}
        & \begin{cases}
            a \in xH \implies \exists h_{1} \in H : a = xh_{1}\\
            a \in yH \implies \exists h_{2} \in H : a = yh_{2}
        \end{cases} \implies xh_{1} = yh_{2} \iff\\
        & xh_{1}\cdot h_{1}^{-1} = yh_{2}h_1^{-1} \iff x = yh_{2}h_{1}^{-1} = yh_{3}, \; h_{3} \in H \implies x \in yH
        \end{split}
    \end{equation}
    Utilizando el lema $\ref{lem:clases-laterales}$, sabemos que $x \in yH \implies xH = yH$.
\end{proof}

\begin{thm}[Teorema de Lagrange]
    \label{thm:lagrange}
    El orden del grupo es múltipo del orden del subgrupo
    \begin{equation}
        H \subseteq G \implies |G| = |H| \cdot n \quad n \in \mathbb{N}
    \end{equation}
\end{thm}

\begin{proof}
    Dado un grupo $G$ con un subgrupo $H$ y con clases laterales $x_iH$, por el teorema \ref{thm:clases-laterales} sabemos que las clases laterales no tienen intersección. Como tienen que abarcar a todo el grupo $G$ y no tienen intersección, entonces necesariamente el orden de $G$ será múltiplo del orden de las clases laterales.
    \begin{equation}
        \sum_{i=1}^{\text{índice}} |x_iH| = |G| \implies |G| = |x_iH| \cdot n \quad n \in \mathbb{N}
    \end{equation}
    Las clases laterales $x_iH$ tienen el mismo orden que $H$, ya que se construyen operando $x_i$ con los elementos de $H$, entonces el orden de $G$ es múltiplo del orden de $H$.
    \begin{equation}
        |x_iH| = |H| \implies |G| = |H| \cdot n \quad n \in \mathbb{N}
    \end{equation}
\end{proof}

\begin{cor}
    El orden del grupo es múltiplo del orden de un elemento
    \begin{equation}
        x \in G: x^{k} = e \implies |G| = k \cdot n \quad n \in \mathbb{N}
    \end{equation}
    \label{cor:lagrange}
\end{cor}

\begin{thm}
    Cualquier grupo de orden primo es cíclico
    \begin{equation}
        |G| = p \text{ primo} \implies \exists x \in G : x^{p} = e 
    \end{equation}
\end{thm}

\begin{proof}
    Por el colorario \ref{cor:lagrange} sabemos que el orden del grupo tiene que ser múltiplo del orden de los elementos. Como el orden del grupo es primo, los elementos solo pueden tener orden $1$ u orden $p$. Por el teorema \ref{thm:neutro}, el único elemento de orden 1 es el neutro, entonces el resto de elementos del grupo tienen orden $p$. Como existe al menos un elemento de orden $p$, el grupo es cíclico.
\end{proof}

\begin{thm}
    Si $H \subset G$ es un subgrupo, entonces $gHg^{-1}$, con $g \in G$, también lo es.
    \begin{equation}
        H \subset G \implies gHg^{-1} \subset G \quad g \in G 
    \end{equation}
\end{thm}

\begin{proof}
    $gHg^{-1}$ cumple con las tres propiedades del los subgrupos:
    \begin{enumerate}
        \item $gHg^{-1}$ seguirá manteniendo la propiedad asociativa ya que solo hemos operado los elementos de $H$ con elementos de $G$.
        \item El elemento neutro $e\in G$ está en $gHg^{-1}$:
        \begin{equation}
            e \in H \implies g \cdot e \cdot g^{-1} = g \cdot g^{-1} = e \implies e \in gHg^{-1}
        \end{equation}
    \item Cada elemento $x = ghg^{-1} \in gHg^{-1}$ tiene inverso $x^{-1}$:
        \begin{equation}
            ghg^{-1} \cdot g(h^{-1})g^{-1}    = gh (h^{-1})g^{-1} \cdot g^{-1} = g g^{-1} = e \implies g(h^{-1})g = x^{-1} \in gHg^{-1}       
        \end{equation}
    \end{enumerate}
\end{proof}

\begin{thm}
    Para todo grupo $G = \{X, \cdot\}$ con un subgrupo $H \subset G$:
    \begin{equation}
        gHg^{-1} = H \quad g \in G \iff xH = Hx
    \end{equation}

    \begin{proof}
        $\implies$
        \begin{equation}
            gHg^{-1} = H \iff gHg^{-1}\cdot g = Hg \iff gH = Hg \quad h \in G   
        \end{equation}
    \end{proof}

    \begin{proof}
        $\impliedby$
        \begin{equation}
            xH = Hx \implies xH\cdot x^{-1} = Hx\cdot x^{-1} \iff xHx^{-1} = H \quad x \in G     
        \end{equation}
    \end{proof}
\end{thm}

\begin{thm}
    Las clases laterales izquierdas $x_iH$ de un subgrupo $H$ normal, forman un grupo.
\end{thm}

\begin{proof}
    Las clases laterales izquierdas de un subgrupo $H$ cumplen con las tres propiedades de los grupos:
    \begin{enumerate}
        \item La operación no se sale del conjunto y por tanto conserva la propiedad asociativa:
            \begin{equation}
                xH = Hx \implies \forall a \in xH : xH \cdot yH = xyHH = xyH \implies ab \in xyH
            \end{equation}
        \item El neutro pertenece a la clase lateral izquierda $eH$.
        \item Todo elemento $a \in xH$ tiene inverso $a^{-1} \in (x^{-1})H$:
            \begin{equation}
                \forall a = xh_{1} \in xH \; \exists a^{-1} = (x^{-1})h_{2}  \in (x^{-1})H : h_{1}x \cdot (x^{-1})h_2 = h_{1}h_{2} \in eH \quad h_{1},h_{2} \in H
            \end{equation}
    \end{enumerate}
\end{proof}

\begin{thm}
    El conmutador es un subgrupo y es normal.
\end{thm}

\begin{proof}
    El conmutador $G' = \{<[a,b]> : a,b \in G\}$ es un subgrupo.
    \begin{enumerate}
        \item Como estamos operando elementos de $G$ con la misma operación, se seguirá conservando la\\ propiedad asociativa.
        \item El elemento neutro $[a,e] = e$ pertenece a $C$:
            \begin{equation}
                [a,e] = a\cdot e\cdot a^{-1} \cdot e^{-1} = a \cdot e \cdot a^{-1} \cdot e = a \cdot a^{-1} = e \implies e \in C \quad a,b \in G
            \end{equation}
        \item Todo elemento $x = [a,b]$ tiene inverso $x^{-1} = [b,a]$
            \begin{equation}
                \begin{split}
                    & \forall x = [a,b] = aba^{-1}b^{-1} \in C \exists x^{-1} = [b,a] = bab^{-1}a^{-1} \in C : x \cdot x^{-1} =\\
                    & aba^{-1}b^{-1} \cdot bab^{-1}a^{-1} = aba^{-1}ab^{-1}a^{-1} = abb^{-1}a^{-1} = aa^{-1} = e\\   
                \end{split}
            \end{equation}
    \end{enumerate}
\end{proof}

\begin{proof}
    El conmutador $C = <[a,b]>$ es un subgrupo normal.
    \begin{equation}
        \forall h \in G' : aha^{-1} = aha^{-1}h^{-1}h = [a, h] \cdot h \in G' \implies g(G')g = G'
    \end{equation}
\end{proof}

\begin{thm}
    El núcleo de un homomorfismo es un subgrupo y es normal.
\end{thm}

\begin{proof}
    La operación está contenida en $\text{Ker}f$:
    \begin{equation}
        \forall a,b \in \text{Ker} f: f(a\cdot b) = f(a) \cdot f(b) = e \cdot e = e \implies a\cdot b \in \text{Ker} f
    \end{equation}

    \begin{enumerate}
        \item El subgrupo del nucleo al mantener la operación, sigue manteniendo también la propiedad asociativa.
        \item El neutro pertenece al nucleo:
        \begin{equation}
            f(e) = e \implies e \in \text{Ker} f
        \end{equation}
        \item Si $x \in \text{Ker} f$ entonces su inverso también.
        \begin{equation}
            \begin{split}
                & x \in \text{Ker} f : f(e) = e \implies f(x \cdot x^{-1}) = e \implies f(x) \cdot f(x^{-1}) = e \implies\\
                & e \cdot f(x^{-1}) = e \implies f(x^{-1}) = e \implies x^{-1} \in \text{Ker} f 
            \end{split}
        \end{equation}
    \end{enumerate}

    Por lo tanto, el nucleo es un homomorfismo.

    Además, es normal ya si llamamos $H = \text{Ker} f$ que se cumple que $\forall g \in G: g H g^{-1} = H$.
    \begin{equation}
        f(gHg^{-1}) = f(g) \cdot f(H) \cdot f(g^{-1}) = f(g) \cdot f(g^{-1}) = f(g) \cdot f^{-1}(g) = e \implies g H g^{-1} = H  
    \end{equation}
\end{proof}

\begin{thm}
    La imagen de un homomorfismo es un subgrupo.
\end{thm}

\begin{proof}
    La operación está contenida en $\text{Im} f$:
    \begin{equation}
        f(a) \cdot f(b) = f(a\cdot b) \in \text{Im} f
    \end{equation}

    \begin{enumerate}
        \item Como se conserva la operación, se sigue conservando la propiedad asociativa.
        \item Como $f(e) = e$, el neutro pertenece a $\text{Im} f$.
        \begin{equation}
            f(e) = e \implies e \in \text{Im} f
        \end{equation}
        \item Si $x = f(a) \in \text{Im} f$, entonces $x^{-1} = f(a^{-1})$:
        \begin{equation}
            x \cdot x^{-1} = f(a) \cdot f(a^{-1}) = f(a\cdot a^{-1}) = f(e) = e
        \end{equation}
    \end{enumerate}
\end{proof}


\begin{thm}
    La aplicación $G \to G / H:\pi(g) = gH$ es un homomorfismo suprayectivo.
\end{thm}

\begin{proof}
    La aplicación $G \to G / H:\pi(g) =gH$ es un homomorfismo:
    \begin{equation}
        \pi(g_{1}) \cdot \pi (g_{2}) = g_{1}H \cdot g_{2}H = g_{1}g_{2}H H = g_{1}g_{2}H = \pi(g_{1}\cdot g_{2})
    \end{equation}

    La aplicación $\pi(g)$ es suprayectiva ya que toda imagen $gH$ tiene al menos $g$ como preimagen.
\end{proof}

\begin{thm}[Pimer teorema de isomorfía]
    Sea $f:G \to G'$ un homomorfismo. Entonces podemos establecer el isomorfismo $\bar{f}:G / \text{Ker} f \to G' \cong \text{Im}(f)$   
\end{thm}

\begin{proof}
    Renombremos $\text{Ker} f=H$. Conocemos el homorfismo cańonico $\pi : G \to G / H$ $\phi(g) = gH$. El inverso será $\pi_{-1} (gH) = g$ tal que $\pi^{-1}: G / H \to G$. Entonces:
    \begin{equation}
        \pi^{-1}(gH) = g \iff f(\pi^{-1}(gH)) = f(g)
    \end{equation}

    Si a $f \circ \pi^{-1} = \bar{f}$, entonces:
    \begin{equation}
        f(\pi^{-1}(gH)) = f(g) \implies \bar{f}(gH) = f(g) \quad \bar{f}: G / H \to G'
    \end{equation}

    Primero demostremos que es un homomorfismo:
    \begin{equation}
        \bar{f} (aH \cdot bH) = \bar{f} (abHH) = \bar{f}(abH) = f(ab) = f(a) \cdot f(b)
    \end{equation}

    Ahora que es un isomorfismo:
    \begin{itemize}
        \item $\bar{f}$ es sobreyectiva ya que para cualquier imagen $f(a)$ existe la preimagen $aH$.
        \item $\bar{f}$ es inyectiva:

            Queremos demostrar que $aH \neq bH \implies f(a) \neq f(b)$, podemos demostrar que $\neg \bar{f}(aH) \neq \bar{f}(bH) \implies \neg aH \neq bH$, es decir, que $\bar{f}(aH) = \bar{f}(bH) \implies aH = bH$.
        \begin{equation}
            \begin{split}
                & \bar{f}(aH) = \bar{f}(bH) \iff f(a) = f(b) \iff f(a) \cdot f(b^{-1}) = f(b) \cdot f(b^{-1}) \iff\\
                & f(ab^{-1}) = f(bb^{-1}) = f(e) = e \implies ab^{-1} \in H \implies ab^{-1}H = H \iff\\
                & ab^{-1}H\cdot b = H\cdot b \iff ab^{-1}bH = Hb \iff aH = Hb = bH \iff aH = bH   
            \end{split}
        \end{equation}
    \end{itemize}

    Por lo tanto, hemos encontrado un isomorfismo $\bar{f}: G / \text{Ker} f \to G' \cong \text{Im} f$.
\end{proof}

\begin{thm}
    Todo cuerpo es dominio de integridad
    \begin{equation}
        G \text{ cuerpo} \implies G \text{ D.I.}
    \end{equation}
\end{thm}

\begin{proof}
    Supongamos que $G$ es un cuerpo pero que no es un dominio de integridad.

    Entonces como $G$ no es un dominio de integridad se cumple que:
    \begin{equation}
        \exists a, b \in G : ab = 0 \quad a,b \neq 0
    \end{equation}

    Como $G$ es un cuerpo todos sus elementos menos el $0$ tienen inverso multiplicativo, por lo tanto:
    \begin{equation}
        ab \cdot b^{-1}a^{-1} = 1 \implies 0 \cdot (ab)^{-1} = 1 
    \end{equation}

    Pero el inverso aditivo no puede tener inverso multiplicativo ya que $G$ es un cuerpo. Contradicción.
\end{proof}

\begin{thm}
    Un ideal $I$ que posee al neutro multiplicativo coincide con el anillo que lo contiene.
    \begin{equation}
        1 \in I \subseteq R \implies I = R
    \end{equation}
\end{thm}

\begin{proof}
    Como $1 \in I$ entonces, por definición: 
    \begin{equation}
        \forall x \in R: 1 \cdot x \in I \implies \forall x \in R: x \in I \implies I = R
    \end{equation}
\end{proof}

\begin{thm}
    Todos los ideales en $\mathbb{Z}$ son principales.
\end{thm}

\begin{proof}
    Si el ideal es el trivial entonces ya es principal.

    Como los ideales contienen a los inversos aditivos $-a$ de los elementos $a\neq 0$, hablaremos siempre de números positivos.

    Consideremos $m$ como el elemento más pequeño del ideal:

    si $m=1$ el ideal coincide con todo $\mathbb{Z}$ y por lo tanto ya es principal.

    En caso contrario $m\neq 1$, si el ideal no fuese principal, entonces el resto de elementos no serían múltiplos de $m$:
    \begin{equation}
        i \in I: m \mid i \implies i = mk + r \quad r < m \implies r = i - mk
    \end{equation}

    Pero por definición de ideal, como $k \in \mathbb{Z}$, entonces $mk \in I$. Como $I$ es un subgrupo de $\mathbb{Z}$:
    \begin{equation}
        i - mk \in I \implies r \in I
    \end{equation}

    Contradicción. Hemos encontrado un elemento $r \in I$ más pequeño que $m$ cuando hemos dicho que $m$ era el más pequeño. Por lo tanto, hemos demostrado por reducción al absurdo que todos los ideales en $\mathbb{Z}$ son principales.
\end{proof}

\begin{thm}
    En un anillo conmutativo $A$ con un ideal $I$, el anillo cociente $A / I$ es D.I. si y solo si $I$ es un ideal primo.
\end{thm}

\begin{proof}
    $\implies$
    \begin{equation}
        \begin{split}
            & (a+I)(b+I) = ab + I = e + I \implies ab \in I\\
            & A / I \text{ D.I} \implies ab \in I \implies (a + I)(b + I) = ab + I = I \implies\\
            & a + I = I \text{ o } b + I = I \implies a \in I \text{ o } b \in I
        \end{split}
    \end{equation}
    $\impliedby$
    \begin{equation}
        \begin{split}
            & I \text{ primo} : ab \in I \implies a \in I \text{ o } b \in I \iff a \not\in I \text{ y } b \not\in I \implies ab \not\in I \implies\\
            & \forall (a + I) \neq I, (b + I) \neq I : ab + I = (a +I)(b + I) \neq I \implies A / I \text{ D.I.}
        \end{split}
    \end{equation}    
\end{proof}


    \section{Contraejemplos}

\begin{ce}
    Todos los ideales en $\mathbb{Z}[x]$ son principales.
\end{ce}

\begin{proof}
    $\mathbb{Z}[x,2]$ (polinomios con termino independiente par) es un ideal ya que si se multiplica con cualquier otro $P(x) \in \mathbb{Z}[x]$ o el termino independiente es cero o es par (ya que par por impar es par). Pero no es principal ya que tanto $2$ como $x$ tendrían que ser generados por un mismo $P(x)$:
    \begin{equation}
        \begin{cases}
            & 2 = P(x)Q_{1}(x)\\
            & x = P(x)Q_{2}(x)
        \end{cases}
    \end{equation}
\end{proof}

\begin{ce}
    La imagen y el núcleo de un homomorfismo de anillos son ideales.
\end{ce}

\begin{proof}
    Es cierto que el núcleo es siempre un ideal, pero la imagen solo si se trata de un homomorfismo suprayectivo. Si consideramos por ejemplo el homomorfismo $f:\mathbb{R} \to M_{2}(\mathbb{R})$ tal que:
    \begin{equation}
        f(x) = \begin{pmatrix}
            x & 0\\
            0 & 0
        \end{pmatrix} 
    \end{equation}

    Entonces $\text{Im} f$ ya no es un ideal ya que la multiplicación de un elemento de $M_{2}(\mathbb{R})$ con la imágen se puede salir de $\text{Im} f$:
    \begin{equation}
        \begin{pmatrix}
            x & 0\\
            0 & 0
        \end{pmatrix} 
        \cdot
        \begin{pmatrix}
            1 & 1\\
            0 & 0
        \end{pmatrix} =
        \begin{pmatrix} 
            x & x\\
            0 & 0
        \end{pmatrix} \not\in \text{Im}f
    \end{equation}
\end{proof}

\begin{ce}
    Un homomorfismo de anillos siempre pasa el elemento neutro multiplicativo del anillo.
\end{ce}

\begin{proof}
    Consideremos el homomorfismo $f:\mathbb{R} \to M_{2}(\mathbb{R})$:
    \begin{equation}
        f(x) = \begin{pmatrix} x & 0\\ 0 & 0 \end{pmatrix} 
    \end{equation}

    El inverso multiplicativo del espacio de salida $1 \in \mathbb{R}$ pasa al $\begin{pmatrix} 1 & 0\\ 0 & 0 \end{pmatrix} \neq \begin{pmatrix} 1 & 0\\ 0 & 1 \end{pmatrix}$
\end{proof}

\begin{ce}
    Un polinomio de grado $n$ no puede tener más de $n$ raíces.
\end{ce}

\begin{proof}
    En el anillo $\mathbb{Z}_{6}$ el polinomio $P(x)=(x-2)(x-3)$ tiene más de 2 raíces:
    \begin{equation}
        \begin{split}
            & P(0) = (0-2)(0-3) = (-2)(-3) = 6 \equiv 0 \mod 6\\
            & P(2) = (2-2)(2-3) = 0\cdot (-3) = 0\\
            & P(3) = (3-2)(3-3) = 1 \cdot 0 = 0\\
            & P(5) = (5-2)(5-3) = 3 \cdot 2 = 6 \equiv 0 \mod 6
        \end{split}
    \end{equation}
\end{proof}

\begin{ce}
    $\mathbb{Q}[\sqrt{2}]$ es isomorfo a $\mathbb{Q}[\sqrt{5}]$
\end{ce}

\begin{proof}
    Como estamos diciendo que es isomorfo, el neutro multiplicativo se tiene que conservar:
    \begin{equation}
        f(1) = 1 \implies f(a) = a
    \end{equation}

    Entonces, necesariamente $f(\sqrt{2}) = a\sqrt{5}$.
    \begin{equation}
        2 = f(2) = f(\sqrt{2}\sqrt{2}) = f(\sqrt{2})f(\sqrt{2}) = (a\sqrt{5})^2 = 5a^2 \implies a^2 = \frac{2}{5}
    \end{equation}

    Pero, $a^2 \neq \frac{2}{5}$ (misma demostración que $\sqrt{2}$ irracional).
\end{proof}

\begin{ce}
    Si un anillo es un dominio de integridad, el anillo cociente por un ideal también es un D.I.
\end{ce}

\begin{proof}
    Consideremos $\mathbb{Z} / 6\mathbb{Z}$. $\mathbb{Z}$ es un D.I. pero el anillo cociente no:
    \begin{equation}
        (2 + 6\mathbb{Z}) (3 + 6\mathbb{Z}) = 6 + 6\mathbb{Z} = 0 + 6\mathbb{Z}
    \end{equation}
\end{proof}

\begin{ce}
    Todos los subanillos de anillos conmutativos son ideales.
\end{ce}

\begin{proof}
    Consideremos el subanillo $A = \{(n,n): n \in \mathbb{Z}\}$ el cual es subanillo de un anillo conmutativo $A \subset \mathbb{Z} \times \mathbb{Z}$. Este subanillo no es un ideal:
    \begin{equation}
        (n,n) \cdot (1,0) = (n,0) \not\in A
    \end{equation}
\end{proof}

\begin{ce}
    El neutro multiplicativo del subanillo coincide con la del anillo principal.
\end{ce}

\begin{proof}
    $\{0,2,4\} \subset \mathbb{Z}_{6}$ tiene como neutro multiplicativo al $4$:
    \begin{equation}
        \begin{split}
            & 0 \cdot 4 = 0\\
            & 2 \cdot 4 = 8 \equiv 2 \mod 4\\
            & 4 \cdot 4 = 16 \equiv 4 \mod 4
        \end{split}
    \end{equation}
    Pero en $\mathbb{Z}_{6}$ el neutro multiplicativo es el $1$.
\end{proof}

\begin{ce}
    $\mathbb{Z}[\sqrt{-7}]$ es D.F.U.
\end{ce}

\begin{proof}
    El elemento $8$ tiene dos factorizaciones:
    \begin{equation}
        \begin{split}
            & 8 = 2\cdot 2\cdot 2\\
            & 8 = (1 + \sqrt{-7})(1-\sqrt{-7})
        \end{split}
    \end{equation}

    Por lo tanto, hemos encontrado un elemento que no tiene una descomposición única.
\end{proof}

\begin{ce}
    Cualquier dominio de factorización es dominio de factorización única.
\end{ce}

\begin{proof}
    El anillo $\mathbb{Z}[\sqrt{-7}]$ es dominio de factorización ya que los coeficientes de los elementos $a + b\sqrt{-7}$ pueden ser descompuestos en primos.

    Sin embargo, no es dominio de factorización única ya que el elemento $8$ tiene más de una factorización:
    \begin{equation}
        \begin{split}
            & 8 = 2 \cdot 2 \cdot 2\\
            & 8 = (1 + \sqrt{-7})(1 - \sqrt{-7})
        \end{split} 
    \end{equation}
\end{proof}


\end{document}
